% Options for packages loaded elsewhere
\PassOptionsToPackage{unicode}{hyperref}
\PassOptionsToPackage{hyphens}{url}
%
\documentclass[
]{article}
\usepackage{amsmath,amssymb}
\usepackage{iftex}
\ifPDFTeX
  \usepackage[T1]{fontenc}
  \usepackage[utf8]{inputenc}
  \usepackage{textcomp} % provide euro and other symbols
\else % if luatex or xetex
  \usepackage{unicode-math} % this also loads fontspec
  \defaultfontfeatures{Scale=MatchLowercase}
  \defaultfontfeatures[\rmfamily]{Ligatures=TeX,Scale=1}
\fi
\usepackage{lmodern}
\ifPDFTeX\else
  % xetex/luatex font selection
\fi
% Use upquote if available, for straight quotes in verbatim environments
\IfFileExists{upquote.sty}{\usepackage{upquote}}{}
\IfFileExists{microtype.sty}{% use microtype if available
  \usepackage[]{microtype}
  \UseMicrotypeSet[protrusion]{basicmath} % disable protrusion for tt fonts
}{}
\makeatletter
\@ifundefined{KOMAClassName}{% if non-KOMA class
  \IfFileExists{parskip.sty}{%
    \usepackage{parskip}
  }{% else
    \setlength{\parindent}{0pt}
    \setlength{\parskip}{6pt plus 2pt minus 1pt}}
}{% if KOMA class
  \KOMAoptions{parskip=half}}
\makeatother
\usepackage{xcolor}
\usepackage[margin=1in]{geometry}
\usepackage{color}
\usepackage{fancyvrb}
\newcommand{\VerbBar}{|}
\newcommand{\VERB}{\Verb[commandchars=\\\{\}]}
\DefineVerbatimEnvironment{Highlighting}{Verbatim}{commandchars=\\\{\}}
% Add ',fontsize=\small' for more characters per line
\usepackage{framed}
\definecolor{shadecolor}{RGB}{248,248,248}
\newenvironment{Shaded}{\begin{snugshade}}{\end{snugshade}}
\newcommand{\AlertTok}[1]{\textcolor[rgb]{0.94,0.16,0.16}{#1}}
\newcommand{\AnnotationTok}[1]{\textcolor[rgb]{0.56,0.35,0.01}{\textbf{\textit{#1}}}}
\newcommand{\AttributeTok}[1]{\textcolor[rgb]{0.13,0.29,0.53}{#1}}
\newcommand{\BaseNTok}[1]{\textcolor[rgb]{0.00,0.00,0.81}{#1}}
\newcommand{\BuiltInTok}[1]{#1}
\newcommand{\CharTok}[1]{\textcolor[rgb]{0.31,0.60,0.02}{#1}}
\newcommand{\CommentTok}[1]{\textcolor[rgb]{0.56,0.35,0.01}{\textit{#1}}}
\newcommand{\CommentVarTok}[1]{\textcolor[rgb]{0.56,0.35,0.01}{\textbf{\textit{#1}}}}
\newcommand{\ConstantTok}[1]{\textcolor[rgb]{0.56,0.35,0.01}{#1}}
\newcommand{\ControlFlowTok}[1]{\textcolor[rgb]{0.13,0.29,0.53}{\textbf{#1}}}
\newcommand{\DataTypeTok}[1]{\textcolor[rgb]{0.13,0.29,0.53}{#1}}
\newcommand{\DecValTok}[1]{\textcolor[rgb]{0.00,0.00,0.81}{#1}}
\newcommand{\DocumentationTok}[1]{\textcolor[rgb]{0.56,0.35,0.01}{\textbf{\textit{#1}}}}
\newcommand{\ErrorTok}[1]{\textcolor[rgb]{0.64,0.00,0.00}{\textbf{#1}}}
\newcommand{\ExtensionTok}[1]{#1}
\newcommand{\FloatTok}[1]{\textcolor[rgb]{0.00,0.00,0.81}{#1}}
\newcommand{\FunctionTok}[1]{\textcolor[rgb]{0.13,0.29,0.53}{\textbf{#1}}}
\newcommand{\ImportTok}[1]{#1}
\newcommand{\InformationTok}[1]{\textcolor[rgb]{0.56,0.35,0.01}{\textbf{\textit{#1}}}}
\newcommand{\KeywordTok}[1]{\textcolor[rgb]{0.13,0.29,0.53}{\textbf{#1}}}
\newcommand{\NormalTok}[1]{#1}
\newcommand{\OperatorTok}[1]{\textcolor[rgb]{0.81,0.36,0.00}{\textbf{#1}}}
\newcommand{\OtherTok}[1]{\textcolor[rgb]{0.56,0.35,0.01}{#1}}
\newcommand{\PreprocessorTok}[1]{\textcolor[rgb]{0.56,0.35,0.01}{\textit{#1}}}
\newcommand{\RegionMarkerTok}[1]{#1}
\newcommand{\SpecialCharTok}[1]{\textcolor[rgb]{0.81,0.36,0.00}{\textbf{#1}}}
\newcommand{\SpecialStringTok}[1]{\textcolor[rgb]{0.31,0.60,0.02}{#1}}
\newcommand{\StringTok}[1]{\textcolor[rgb]{0.31,0.60,0.02}{#1}}
\newcommand{\VariableTok}[1]{\textcolor[rgb]{0.00,0.00,0.00}{#1}}
\newcommand{\VerbatimStringTok}[1]{\textcolor[rgb]{0.31,0.60,0.02}{#1}}
\newcommand{\WarningTok}[1]{\textcolor[rgb]{0.56,0.35,0.01}{\textbf{\textit{#1}}}}
\usepackage{graphicx}
\makeatletter
\def\maxwidth{\ifdim\Gin@nat@width>\linewidth\linewidth\else\Gin@nat@width\fi}
\def\maxheight{\ifdim\Gin@nat@height>\textheight\textheight\else\Gin@nat@height\fi}
\makeatother
% Scale images if necessary, so that they will not overflow the page
% margins by default, and it is still possible to overwrite the defaults
% using explicit options in \includegraphics[width, height, ...]{}
\setkeys{Gin}{width=\maxwidth,height=\maxheight,keepaspectratio}
% Set default figure placement to htbp
\makeatletter
\def\fps@figure{htbp}
\makeatother
\setlength{\emergencystretch}{3em} % prevent overfull lines
\providecommand{\tightlist}{%
  \setlength{\itemsep}{0pt}\setlength{\parskip}{0pt}}
\setcounter{secnumdepth}{-\maxdimen} % remove section numbering
\ifLuaTeX
  \usepackage{selnolig}  % disable illegal ligatures
\fi
\IfFileExists{bookmark.sty}{\usepackage{bookmark}}{\usepackage{hyperref}}
\IfFileExists{xurl.sty}{\usepackage{xurl}}{} % add URL line breaks if available
\urlstyle{same}
\hypersetup{
  pdftitle={Formative Assignment},
  pdfauthor={202359074},
  hidelinks,
  pdfcreator={LaTeX via pandoc}}

\title{Formative Assignment}
\author{202359074}
\date{7 October 2023}

\begin{document}
\maketitle

This document contains the necessary commands and layout to meet the
formatting requirements for MY472. You should use the template.Rmd file
as the basis for your own answers to the assigned exercises.

\hypertarget{formatting-requirements}{%
\subsection{Formatting requirements}\label{formatting-requirements}}

\begin{itemize}
\item
  You must present all results in full sentences, as you would in a
  report or academic piece of writing

  \begin{itemize}
  \tightlist
  \item
    If the exercise requires generating a table or figure, you should
    include at least one sentence introducing and explaining it. E.g.
    ``The table below reports the counts of Wikipedia articles
    mentioning the LSE, by type of article.''
  \end{itemize}
\item
  Unless stated otherwise, all code used to answer the exercises should
  be included as a code appendix at the end of the script. This
  formatting can be achieved by following the guidance in this template
  file.
\item
  All code should be annotated with comments, to help the marker
  understand what you have done
\item
  Your output should be replicable. Any result/table/figure that cannot
  be traced back to your code will not be marked
\end{itemize}

\hypertarget{example-of-in-line-figures-without-code}{%
\subsection{Example of in-line figures without
code}\label{example-of-in-line-figures-without-code}}

For those interested, we achieve the formatting requirements in
two-steps: 1) in the \texttt{setup} chunk, we set
\texttt{knitr::opts\_chunk\$set(echo\ =\ FALSE)} so that code is not
included (echoed) by default in code chunks; 2) we add a specific chunk
at the end of the file to collect and print \emph{all} the code in the
Rmarkdown file. Do not delete the final code chunk from the template!

For example, below we use a code chunk to generate random data and
include a scatter plot in-line. The code used to generate this chart is
only reported at the end of the document.

\includegraphics{Assignment-1_files/figure-latex/plot_example-1.pdf}

In specific instances, however, you may be directed to report your code
in-line (or you may want to do this to illustrate a specific point). In
these cases, we can override the default behaviour by adding the chunk
option \texttt{echo\ =\ TRUE} to a specific R chunk. When
\texttt{echo=TRUE}, your code is presented in-line with any output
displayed afterwards. The same code will also be included in the
appendix at the bottom of the document (which is fine).

\begin{Shaded}
\begin{Highlighting}[]
\CommentTok{\# \{[language] [chunk\_name], [chunk\_options]\}}
\CommentTok{\# here we use echo=TRUE to override our global options and make the chunk appear exactly here. }

\FunctionTok{print}\NormalTok{(}\StringTok{"This code chunk is visible in this section."}\NormalTok{)}
\end{Highlighting}
\end{Shaded}

\begin{verbatim}
## [1] "This code chunk is visible in this section."
\end{verbatim}

\hypertarget{assignment-1-formative}{%
\section{\texorpdfstring{\textbf{Assignment 1:
Formative}}{Assignment 1: Formative}}\label{assignment-1-formative}}

The following is my submission towards the formative assignment for the
course \textbf{Data for Data Scientist}

\hypertarget{exercise-1-forking-repository}{%
\subsection{\texorpdfstring{\textbf{Exercise 1: Forking
Repository}}{Exercise 1: Forking Repository}}\label{exercise-1-forking-repository}}

The forked version of the assignment template repository can be accessed
from
\emph{\url{https://github.com/snehamariamthomas/472_assignment_template.git}}

\hypertarget{exercise-2-tidy-data}{%
\subsection{\texorpdfstring{\textbf{Exercise 2: Tidy
Data}}{Exercise 2: Tidy Data}}\label{exercise-2-tidy-data}}

For exercise 2, I chose the dataset titled \textbf{USArrests}

2.1. \textbf{About Dataset} It contains the data on violent crime rates
in the United States of America.

2.2. \textbf{Tidy Status} Based on results of `head()', the the crime
rates were spread across three columns : \emph{murder}, \emph{assault}
and \emph{rape}. Hence, the data was had to be converted into long
format.

2.3. \textbf{Tidying Data}

\begin{itemize}
\item
  Firstly, the data frame had state names as rownames. To convert the
  rownames to column, I deployed the tibble package. \emph{Source:
  Wright, Kevin. (2021, November 27). How can I convert row names into
  the first column?. Stack Overflow.
  \url{https://stackoverflow.com/a/29511626} }
\item
  Secondly, i converted the data frame into tibble and transformed it
  from wide format to long format wherein the column represented the
  \emph{State} , \emph{Indicators} and \emph{Statistics} . Consequently,
  the column 2 captured the different types of crime, while column 3
  captured the corresponding rate.
\item
  Thus, the data set has ben tidied as per Hadley Wickham's definition
  wherein each variable has it own column, each observation has its own
  row and each observation has its own cell.
\end{itemize}

\begin{verbatim}
##            Murder Assault UrbanPop Rape
## Alabama      13.2     236       58 21.2
## Alaska       10.0     263       48 44.5
## Arizona       8.1     294       80 31.0
## Arkansas      8.8     190       50 19.5
## California    9.0     276       91 40.6
## Colorado      7.9     204       78 38.7
\end{verbatim}

\begin{verbatim}
## 'data.frame':    50 obs. of  5 variables:
##  $ State   : chr  "Alabama" "Alaska" "Arizona" "Arkansas" ...
##  $ Murder  : num  13.2 10 8.1 8.8 9 7.9 3.3 5.9 15.4 17.4 ...
##  $ Assault : int  236 263 294 190 276 204 110 238 335 211 ...
##  $ UrbanPop: int  58 48 80 50 91 78 77 72 80 60 ...
##  $ Rape    : num  21.2 44.5 31 19.5 40.6 38.7 11.1 15.8 31.9 25.8 ...
\end{verbatim}

\begin{verbatim}
## # A tibble: 6 x 3
##   State   Indicators Statistics
##   <chr>   <chr>           <dbl>
## 1 Alabama Murder           13.2
## 2 Alabama Assault         236  
## 3 Alabama Rape             21.2
## 4 Alabama UrbanPop         58  
## 5 Alaska  Murder           10  
## 6 Alaska  Assault         263
\end{verbatim}

\hypertarget{appendix-all-code-in-this-assignment}{%
\subsection{Appendix: All code in this
assignment}\label{appendix-all-code-in-this-assignment}}

\begin{Shaded}
\begin{Highlighting}[]
\CommentTok{\# this chunk contains code that sets global options for the entire .Rmd. }
\CommentTok{\# we use include=FALSE to suppress it from the top of the document, but it will still appear in the appendix. }

\NormalTok{knitr}\SpecialCharTok{::}\NormalTok{opts\_chunk}\SpecialCharTok{$}\FunctionTok{set}\NormalTok{(}\AttributeTok{echo =} \ConstantTok{FALSE}\NormalTok{) }\CommentTok{\# actually set the global chunk options. }
\CommentTok{\# we set echo=FALSE to suppress code such that it by default does not appear throughout the document. }
\CommentTok{\# note: this is different from .Rmd default}
\FunctionTok{set.seed}\NormalTok{(}\DecValTok{89}\NormalTok{) }\CommentTok{\# set a seed for R\textquotesingle{}s psuedo{-}randomiser, for replicability.}
\NormalTok{x }\OtherTok{\textless{}{-}} \FunctionTok{rnorm}\NormalTok{(}\DecValTok{100}\NormalTok{) }\CommentTok{\# randomly draw 100 obs from normal distribution, save as object}
\NormalTok{y }\OtherTok{\textless{}{-}} \FunctionTok{rnorm}\NormalTok{(}\DecValTok{100}\NormalTok{) }
\FunctionTok{plot}\NormalTok{(x,y) }\CommentTok{\# two{-}way scatterplot using R\textquotesingle{}s default plotting}
\CommentTok{\# \{[language] [chunk\_name], [chunk\_options]\}}
\CommentTok{\# here we use echo=TRUE to override our global options and make the chunk appear exactly here. }

\FunctionTok{print}\NormalTok{(}\StringTok{"This code chunk is visible in this section."}\NormalTok{)}
\FunctionTok{library}\NormalTok{(datasets)}
\FunctionTok{library}\NormalTok{(tibble) }\CommentTok{\#loading package}
\FunctionTok{library}\NormalTok{ (tidyr) }\CommentTok{\#loading package}
\FunctionTok{data}\NormalTok{(USArrests) }\CommentTok{\#loading dataset in R}
\FunctionTok{head}\NormalTok{(USArrests) }\CommentTok{\# Reviewing data}
\NormalTok{USArrests }\OtherTok{\textless{}{-}} \FunctionTok{rownames\_to\_column}\NormalTok{(USArrests, }\AttributeTok{var =} \StringTok{"State"}\NormalTok{)}\CommentTok{\#Converting rownames to column}
\FunctionTok{str}\NormalTok{(USArrests)}

\CommentTok{\#Tidying Data}
\NormalTok{USArrests\_tible }\OtherTok{\textless{}{-}} \FunctionTok{as\_tibble}\NormalTok{(USArrests)}\CommentTok{\#Converting data frame into tibble }
\NormalTok{USArrests\_tidy }\OtherTok{\textless{}{-}}\NormalTok{ USArrests\_tible }\SpecialCharTok{\%\textgreater{}\%}
  \FunctionTok{pivot\_longer}\NormalTok{ (}\FunctionTok{c}\NormalTok{(}\StringTok{\textquotesingle{}Murder\textquotesingle{}}\NormalTok{,}\StringTok{\textquotesingle{}Assault\textquotesingle{}}\NormalTok{,}\StringTok{"Rape"}\NormalTok{,}\StringTok{"UrbanPop"}\NormalTok{), }\AttributeTok{names\_to =} \StringTok{"Indicators"}\NormalTok{, }\AttributeTok{values\_to=}\StringTok{"Statistics"}\NormalTok{) }\CommentTok{\#Data tidying}
\FunctionTok{head}\NormalTok{(USArrests\_tidy)}
\CommentTok{\# this chunk generates the complete code appendix. }
\CommentTok{\# eval=FALSE tells R not to run (\textasciigrave{}\textasciigrave{}evaluate\textquotesingle{}\textquotesingle{}) the code here (it was already run before).}
\end{Highlighting}
\end{Shaded}


\end{document}
